\documentclass[10pt,a4paper,draft]{article}
\usepackage[utf8]{inputenc}
\usepackage{amsmath}
\usepackage{amsfonts}
\usepackage{amssymb}
\usepackage[final]{listings}
\lstset{frame=tb,
  language=Java,
  aboveskip=3mm,
  belowskip=3mm,
  showstringspaces=false,
  columns=flexible,
  basicstyle={\small\ttfamily},
  numbers=none,
  breaklines=true,
  breakatwhitespace=true,
  tabsize=3
}
\author{Vishal Kedia}
\usepackage{geometry}
\geometry{margin=2cm}
\title{Dynamic Programing}
\begin{document}
\maketitle
\section{What is Dynamic Programming?}
\emph{Dynamic Programming} is a technique to solve optimization problems which may have multiple solutions, and the way it works is very similar to that of the divide and conquer technique where in we first solve the sub-problems first and then combine the solutions of those sub problems to find the solution for the actual problem. 

When applying a \emph{Dynamic Programming} to a problem we mainly follow basic following steps.

\begin{enumerate}
\item Characterize the structure of an optimal solution.
\item Recursively define the value of an optimal solution.
\item Compute the value of an \emph{optimal solution}. Typically in a bottom up fashion. 
\item Construct an \emph{optimal solution} from computed information.
\end{enumerate}
\section{Elements of Dynamic Programming}
\subsection{Optimal Substructure}
The first step in solving an problem by \emph{Dynamic Programming} is to characterize the structure of an \textit{optimal solution}. And we build an optimal solution to the problem from optimal solutions to the subproblems.\par You will find the following pattern in determining the optimal substructure:
\begin{enumerate}
\item You show that solution to the problem consists of making a choice, such as choosing an initial cut in the rod.
\item You suppose that for a given problem, you are given the choice that leads to an optimal solution. You do not concern yourself yet with how to determine this choice. You just assume that it has been given to you.
\item Given this choice, you determine which subproblems ensue and how to best characterize the resulting space of subproblems.
\end{enumerate}\par
To characterize the space of the subproblems, a good rule of thumb says to try to keep the space as simple as possible and then expand it as necessary. For example, the space of subproblems that we considered for the \emph{rod cutting problem} contained the problems of optimally cutting a rod of length $i \forall i \in [1..n]$.
\par
Optimal substructure varies across problem domains in two ways:
\begin{enumerate}
\item How many subproblems an optimal solution to the original problem uses, and
\item How many choices we have in determining which subproblem(s) to use in an optimal solution.
\end{enumerate}
\section{Sample Dynamic Programming Problems.}
\subsection{Rod Cutting Problem.}
\textbf{Problem Statment -} Given a \emph{rod} of length \emph{n} inches and a table of prices $p_i$ for $i = 1,2,.....,n$, determine the maximum revenue $r_n$ obtainable by cutting up the rod and selling pieces.
\\[12pt]
\textbf{Discussion -} We can cut the rod in $2^{n-1}$ different ways, since we have independent choise of either cutting or not cutting, at distance \textit{i} inches from the left hand. If the optimal solution cuts the rod into $k$ pieces, for some $1 \geq k \geq n$ then the optimal decomposition will be
\begin{center}
$n = i_1+i_2+...+i_k$
\end{center}
of the rod into pieces of the length $i_1,i_2,..,i_k$ provides maximum corresponding revenue
\begin{center}
$r_n = p_{i_1}+p_{i_2}+...+p_{i_k}$
\end{center}
more generally we can frame the values $r_n$ for $n \geq 1$ in terms of optimal revenues from the shorter rods.
\begin{center}
$r_n = max(p_n,r_1+r_{n-1},r_2+r_{n-2},...,r_{n-1}+r_1)$
\end{center}
Note that here in order to find the optimal solution we are exploring all the choices for making the first cut, however at the time of doing so we are assuming that we know left hand optimal solution for the left hand side of the rod and the right hand side of the rod after makng the first cut, and then choosing the best solution out of all these solutions. Hence we say that \emph{Rod cutting problem} exhibts \textbf{\emph{Optimal Substructure:}} ie. optimal solution to a problem incorporates optimal solutions to related subproblems, which we may solve independently.
\par
In a related but slightly simpler way to arrange a recursive structure of the \emph{Rod Cutting Problem}, we view a decomposition as consistiong of a first piece of length $i$ cut off the left hand end, and then a right hand remainder of length $n-1$. Only the remainder and not the first piece, may be further divided.
\begin{center}
$r_n = \max\limits_{1 \geq i \geq n}(p_i+r_{n-1})$
\end{center}
\subsection{0/1 Knapsack Problem}
\textbf{Problem statment -} Given a list of $n$ items with weights $w_i$ and price $p_i$ for the $i^{th}$ item in the list, and a knapsack of total weight capacity $W$ is given. Determine the maximum value that can be carried at a time in the knapsack.
\\[12pt]
\textbf{Discussion -} We can choose the items in $k$ ways where $k \leq 2^{n-1}$. A solution can be obtained by making the sequence of decision on the items $i_1,i_2,...,i_n$. Suppose we start from the last item then in that case we have two choices that we can make for item $i_n$, which will leave in two possible states, ie. knapsack of capacity $W$ and $i_1,i_2,...,i_{n-1}$ items to take further decision on, or knapsack of capacity $W-w_n$ and $i_1,i_2,...,i_{n-1}$ items to take further decision on. Here we see that making a choice leads us to the smaller subprobles of similar characteristic and hence we can view the decomposition for the optimal solution as
\begin{center}
$f_n(W) = \max(f_{n-1}(W),f_{n-1}(W-w_n))$
\end{center}
\subsection{Kadane's Algorithm - Maximum subarray problem}
\textbf{Problem Statement -} Given an array $A = {a_1,a_2,...,a_n}$ of integers (consisting of both negative and positive values), we need to find the contiguous subarray $A_{i,j}$ where $0 \geq i \geq j \geq n$ such that
\begin{center}
$S_{i,j} = \sum\limits_{k=i}^ja_k$
\end{center}
is maximum for all possible subarrays.
\\[12pt]
\textbf{Discussion -} Suppose we are given maximum for subarray ending at $a_i$ element with sum $S_i$, then what shall be the maximum sum for the subarray ending at element $a_{i+1}$, if we think about it, we see that adding the next element either increases the sum or decreases the sum, so basically we have two choices here either to include it or not, which actually leads us into two states ie. max sum is either $S_{i+1} = S_i + a_{i+1}$ or $S_{i+1} = a_{i+1}$. Here we can say that if the sum increases then we include the element in the subarray and proceed for the next index, or we start afresh the new subarray with only next element in the array and proceed to the next index. So we can view the decomposition as
\begin{center}
$S_i = \max(S_{i-1}+ a_i,a_i)$
\end{center}
Observing the above deduction we can also say that we start the process at index $i=0$ and $S_{-1} = -\infty$ uptill $i = n$ then we shall have our optimal solution $S_n$.
\subsection{Change makning problem - 1}
\textbf{Problem Statement -} A person has to give amount $A$ to another person and he has money in denominations $D = \{d_1,d_2,...,d_k\}$ with infinite count, find the least no of change in which he may pay his due.
\\[12pt]
\textbf{Discussion -} Change making problem is very similar to the knapsack problem, consider the amount to be paid as analogous to the capacity of the knapsack, but the availability of the coins as opposed to the items in knapsack problem never exhausts, and here optimization is in the terms of no of the coins required instead of the maximum value of the items in the knapsack. So in a similar way as that of knapsack, an solution may be obtained by making a sequence of decision on denominations $d_1,d_2,...,d_k$, Suppose we start from the last denominations in that case we have two choices that we can make for the $d_k$, which will leave in two possible states ie. optimal solution either includes atleast one unit or no unit of the $d_k$ denomination. And the minimum of the two subproblems give us the optimal solution. So we can view the decomposition as
\begin{center}
$f_k(A)=\min(f_k(A-d_k)+1,f_{k-1}(A))$
\end{center}
\begin{lstlisting}[language=Java]
	public static int getMinChange(int[] coins, int amt) {
		return getMinChangeInternal(coins,amt,0,coins.length-1);
	}

	private static int getMinChangeInternal(int[] coins, int n, int count,int k) {
		if(n < 0 || k < 0){
			return Integer.MAX_VALUE;//to signify no solution
		}else if(n==0){
			return count;
		}else{
			return Math.min(getMinChangeInternal(coins,n-coins[k],count+1,k), getMinChangeInternal(coins,n,count,k-1));
		}
	}
\end{lstlisting}
\subsection{Change making problem - 2}
\textbf{Problem Statement -} A person has to give amount $A$ to another person and he has money in denominations $D = \{d_1,d_2,...,d_k\}$ with infinite count, find in how many ways he may make the change to pay his due.
\\[12pt]
\textbf{Discussion -} The only difference between the previous problem statement and this one is instead of finding least count solution we are interested in the total no possible solutions, so as we have seen earlier an solution may be obtained by making a sequence of decision on denominations $d_1,d_2,...,d_k$, Suppose we start from the last denominations in that case we have two choices that we can make for the $d_k$, ie. either it contributes to the change or not so the solution that we are interested in will include the sum of the solutions of both subproblems as a result of the two possible decision.
\begin{center}
$f_k(A)=f_k(A-d_k) + f_{k-1}(A)$
\end{center}
\begin{lstlisting}
	public static int getTotalPossibleSolution(int[] coins, int amt) {
		return getTotalPossibleSolutionInternal(coins,amt,coins.length-1);
	}

	private static int getTotalPossibleSolutionInternal(int[] coins, int n,int k) {
		if(n < 0 || k < 0){
			return 0;
		}else if(n==0){
			return 1;
		}else{
			return getTotalPossibleSolutionInternal(coins,n-coins[k],k) + getTotalPossibleSolutionInternal(coins,n,k-1);
		}
	}
\end{lstlisting}
\subsection{Longest Common Subsequence Problem}
\textbf{Problem Statement -} In the \emph{Longest Common Subsequence} problem we are given two sequences $X=\{x_1,x_2,...,x_m\}$ and $Y=\{y_1,y_2,...,y_n\}$ and wish to find a maximum length common subsequence of $X$ and $Y$.
\\[12pt]
\textbf{Discussion -} Let  $X=\{x_1,x_2,...,x_m\}$ and $Y=\{y_1,y_2,...,y_n\}$ be the sequences, and let $Z=\{z_1,z_2,...,z_k\}$ be any LCS of $X$ and $Y$.
\begin{enumerate}
\item if $x_m=y_n$, then $z_k=x_m=y_n$ and $Z_{k-1}$ is an LCS of $X_{m-1}$ and $Y_{n-1}$.
\item if $x_m \neq y_n$, then $z_k \neq x_m$ implies that $Z$ is an LCS of $X_{m-1}$ and $Y$.
\item if $x_m \neq y_n$, then $z_k \neq y_n$ implies that $Z$ is an LCS of $X$ and $Y_{n-1}$.
\end{enumerate}
\subsection{Longest Increasing Subsequences Problem} 
\textbf{Problem Statement -} Given an array of integers, find the longest increasing subsequence in the array. for example say given array is ${50, 3, 10, 7, 40, 80}$ then the longest increasing subsequence is ${3, 7, 40, 80}$.
\\[12pt]
\textbf{Discussion -} In the above problem we can observe that its little similar to the maximum subarray problem, so following the kadane's logic lets assume we have the optimum solution at ending at element $a_i$, then what shall be the longest subsequence ending at element $a_{i+1}$, if we think about it then we can see that if we include the element in the solution in that case solution increases by count $1$ but there is a catch which is if $a_{i+1}$ is in the solution then all the elements befor that needs to be smaller than $a_{i+1}$. So we can view the decomposition as
\begin{center}
$f_A(a_1,a_2,...,a_n) = \max(f_A(a_1,a_2...a_{n-1}),1+f_B(b_1,b_2,...,b_k))$ 
\end{center}
where $B \subset A$ end $b_i \leq a_n$ where $i \in [1,k]$. This approach gives $O(n^2)$ time complexity.
\\[12pt]
This problem can also be solved in order of $O(nlogn)$, in order to solve this we will use the list data-structure with nodes having two data points $(x,c)$, where $x$ represents the last element in the subsequence and $c$ represents the corresponding length of the subsequence, and each node in the list represents a subsequence. Now lets start with the first element in the sequence, we can see that there is no existing subsequence in the list, hence we add the first subsequence in the list with one element so list becomes
\begin{center}
$(50,1)$
\end{center}
now consider the second element $3$, we can see that the $3$ is smaller than the last element of the only subsequence in the list so it will not contribute to the existing subsequence in the list, instead it will form its own subsequence ending at element $3$, hence list may become
\begin{center}
$(50,1) \rightarrow (3,1)$
\end{center}
now there are three possiblility with the next element, it can be either bigger than $50$ then it will be bigger than $3$ also as $3 \leq 50$, hence in this case length for both the sequences, if the next element is between $3$ and $50$ then in that case length of the second sequence increases by one compared to first subsequence, and in the third case when the next element is smaller than both the elements in that case it doesn't contributes to either of the subsequences, so we can conclude that whatever be the next element best solution lies with the second subsequence only, hence instead of adding the second subsequence in the list we shall replace it with the new subsequence as  
\begin{center}
$(50,1) \overset{3}{\Longrightarrow} (3,1)$
\end{center}
now if we consider the next element $10$ then we see that it contributes to the existing subsequence in the list hence the list becomes
\begin{center}
$(3,1) \overset{10}{\Longrightarrow} (10,2)$
\end{center}
when we proceed to the next element $7$, then we see that it is smaller than the last element in the subsequence, so as we have done earlier where we replaced the subsequence ending with bigger element to the subsequence ending with the smaller one, we can't do the same in this case because if the element is inbetween the two elements it will definitely contribute to the subsequence ending with the lower element but still in that case length becomes equals only. So we can conclude that will all possiblility of next element best solutions can be either of the solutions so in this case we will have to mantain both the subsequences
\begin{center}
$(10,2) \overset{7}{\Longrightarrow} (10,2) \rightarrow (7,1)$
\end{center}
now lets proceed to the next element $40$ which happens to be the greater than the ending element in both the subsequences in the list, hence will contribute to both the lists and becomes the last element of both the subsequences, but with different sizes, hence we don't need to keep both the subsequences, we can just keep that subsequence which is bigger hence
\begin{center}
$(10,2) \rightarrow (7,1) \overset{40}{\Longrightarrow} (40,3)$
\end{center}
now lets proceed to the next element $80$, and we see that it is greater than the last element of the subsequence hence contributes to the existing subsequence in the list
\begin{center}
$(40,3) \overset{80}{\Longrightarrow} (80,4)$
\end{center}
\subsection{Longest Palindrome Subsequence Problem}
\textbf{Problem Statement -} Given an efficient algorithm to find the longest \emph{palindrome} that is the subsequence of a given input string.
\\[12pt]
\textbf{Discussion -} If we think about it, this problem is very similar to \emph{Longest Common Subsequence} problem, Let say we are given a sequence $X = \{x_1,x_2,...,x_n\}$, then at any given situation we have a subsequence $X_{i,j} = \{x_i,x_{i+1},....x_j\}$ such that $0 \leq i \leq j \leq n$, and $Z = \{z_1,z_2,...,z_p,...,z_{k-1},z_k\}$ be the \emph{Longest Pandrome Subsequence} then we have following conditions
\begin{enumerate}
\item if $x_1=x_n$, then $z_k=z_1=x_n=x_1$, and $Z_{2,k-1}$ is the solution of $X_{2,n-1}$ 
\item if $x_1 \neq x_n$, then $(z_k = z_1) \neq x_n$ implies that $Z$ is the solution of $X_{1,n-1}$ 
\item if $x_1 \neq x_n$, then $(z_k = z_1) \neq x_1$ implies that $Z$ is the solution of $X_{2,n}$ 
\end{enumerate}
\subsection{Printing Neatly Problem}
\textbf{Problem Statement -} Consider the problem of neatly printing a paragraph with monospaced font (all charachters having the same width) on a printer. The input text is the sequence of $n$ words of length $l_1,l_2,...,l_n$ measured in characters. We want to print this paragraph neatly on a number of lines that hold a maximum of $M$ characters each. Our criterion of \emph{neatness} is as follows. If a given line contains words $i$ through $j$, where $i \leq j$, and we leave exactly one space between words, the number of extra space characters between the lines is
\begin{center}
$p_{i,j}=M-j+i-\sum\limits_{k=i}^jl_k$
\end{center}
\textbf{Discussion -} Suppose that we say that the minimum cost is $f(i,j)$ for accomodating the words from $i$ through $j$. Then decision that we have to take is how many words are going to be present in the first row, in that case our problem in decomposed view looks like
\begin{center}
$f(1,n)=\min(p_{1,k-1}+f(k,n),p_{1,k}+f(k+1,n))$ where $1 < k \leq n : p_{1,k} \geq 0$
\end{center}
\end{document}
